
\documentclass[twocolumn]{article}
\usepackage[top=0.5in, left=0.45in, right=0.45in, bottom=0.5in]{geometry}

\usepackage{url}
% \usepackage{code}
% \usepackage{cite}
\usepackage{amsmath}
\usepackage{amssymb}
\usepackage{graphicx}

\usepackage{nopageno}

\interfootnotelinepenalty=0

% lets me explicitly set a. or 1. etc. as enum label
\usepackage{enumitem}

\pagestyle{empty}

\usepackage{ulem}
% go back to italics for emphasis, though
\normalem

\usepackage{natbib}

\setlength{\footnotesep}{2em}

\newcommand\comment[1]{}
\newcommand\sfrac[2]{\!{}\,^{#1}\!/{}\!_{#2}}

\newcommand\nan{\textsf{NaN}}
\renewcommand\inf{\textsf{inf}}

\newcommand\plusminus{\pm}

\begin{document} 

% flInf flops?
\title{NaN gates and flip FLOPS}
\author{Dr.~Tom~Murphy~VII~Ph.D.\footnote{
Copyright \copyright\ 2019 the Regents of the Wikiplia
Foundation. Appears in SIGBOVIK 2019 with the
XXX JOKE
of the Association for Computational Heresy; 
{\em IEEEEEE!} press, Verlag-Verlag volume no.~0x40-2A.
\$-0.00 } }

\renewcommand\th{\ensuremath{{}^{\textrm{th}}}}
\newcommand\st{\ensuremath{{}^{\textrm{st}}}}
\newcommand\rd{\ensuremath{{}^{\textrm{rd}}}}
\newcommand\nd{\ensuremath{{}^{\textrm{nd}}}}
\newcommand\at{\ensuremath{\scriptstyle @}}

\date{1 April 2019}

\maketitle \thispagestyle{empty}

\begin{abstract}
what do I put here
\end{abstract}

\section*{Introduction}

Mathematics is fundamental to computer science, and the foundation of
mathematics is the real numbers; this is obvious from the name. One of
computing's dirtiest secrets, however, is that computers themselves
are not based on real numbers---rather, they are based on so-called
``ones'' and ``zeroes'' combined with ``logic gates'' simulated with
transistors. While this suffices for most practical purposes, it is
unsatisfying from a theoretical perspective.

Recently, some progress has been made by human geniuses on completely
replacing integer calculations with calculations on real
numbers\cite{fluint8}. While this removes many of the hacks present in
modern software, there are still many components of the computer (e.g.
RAM, registers, the {\it scroll lock} LED, a tiny USB-powered fan that
can cool you on hot summer days or during particularly strenuous
programming sessions) % XXX cite skymall etc?
that are not integer-based.

In this paper I give a new foundation for computing based solely on
real numbers. This ...

I begin with a brief reminder of the definition of real numbers,
although the reader is expected to be familiar as these are pretty
fundamental to everything. The approach of the paper is then to
identify a pair of real numbers that have nice properties
(Section~\ref{sec:distinguished}), and then to give mathematical
operations on these numbers that parallel the logical operations
typically used in the construction of computers
(Section~\ref{sec:logical}). I then discuss how these operations
can be implemented efficiently (Section~\ref{sec:representations}).
I conclude with some wild speculation.

\section{Real numbers}
The real numbers are described by IEEE 754, most recently revised in
AD 2008\cite{ieee754}. Every real number has a sign, a mantissa, and an
exponent. Actually, this understates the elegance of real numbers,
since there are a number of numbers, such as \nan\ (``not a number'')
which are not of this form; \nan\ nas no sign nor mantissa nor
exponent. We also have \inf\ and $-\inf$, which do have a sign, but no
mantissa nor exponent. These are the infinite numbers that you get if
you count very high or very low. Excitingly, we also have both
positive and negative versions of 0. Some numbers have multiple
representations, and most numbers cannot be represented at all.

The real numbers have an equality operation {\tt ==}. This operation
has some very exciting properties: It is not reflexive (\nan\ {\tt ==}
\nan\ is false), and does not obey substitution (for $+0$ {\tt ==} $-0$
is true, but $\sfrac{1}{+0}$ {\tt ==} $\sfrac{1}{-0}$ is false).

Real numbers are an absolute joy to work with.


\subsection{Choosing some distinguished values} \label{sec:distinguished}



use nan and inf because other real numbers can be ``easily confused''
with one another. And do we really want ordering relationships on
our fundamental particles, like $0 < 1$?


% really lean into the joke that we choose the "number" NAN


ha okay now there is just notes

Some useful primitives:

        These are called minNum and maxNum in IEE 754-2008 (5.3.1, p19)

        "If exactly one argument is NaN, they return the other.
        If both are NaN they return NaN.

        NAND  0 1     MAX  nan inf
             +---         +-------
           0 |1 1     nan |nan inf
           1 |1 0     inf |inf inf

                    HYPOT  nan inf
                          +-------
                      nan |nan inf
                      inf |inf inf

        So if 0 = nan and 1 = inf, then MAX is OR.

        We also have AND, since only inf*inf (aka 1 AND 1)
        returns inf in that truth table.

        Is AND and OR together complete? (I don't think so?)
        Do we have NOT?


YES!

Here are some:
Min(-x, -1.0) + inf
hypot(nan, max(1/x, -inf))
inf - max(x, 1)
sqrt(copysign(inf, -x))


Now direct search for some gates:
NAND: inf - max(x + y, -inf)
AND: x + y
OR: hypot(x, y)
XOR: Abs(max(y, -inf) - max(x, -inf))
     Abs(min(y, -x) + max(y, -inf))
     (Wouldn't be surprised if there's a simpler one,
     but my search does not find it before it gets
     out of hand...)
    
NOR: -inf / max(y, max(x, -1))
     


And how about if we leave out min, max? They are certainly
reasonable since they are defined in IEEE 754, but perhaps
not the best joke since they are not as math-y as e.g. hypot


NOR: sqrt(copysign(-inf, -hypot(y, x)))
(Here, hypot is just computing OR.)

So another NOT is:
sqrt(copysign(-inf, -x))  (actually can be inf)

XOR: abs(copysign(-inf, y) - copysign(-inf, x))
OR: hypot(y, x)
AND: x + y
NAND: inf - copysign(-inf, x + y)


ok cool, but copysign kinda has the same problems as min,max.
maybe without that?


the database saturates with 36 entries if we remove
copysign and min, max. They are these:

[nan, nan, nan, nan] = NaN
[nan, nan, nan, ~inf] = Minus(Neg(y), x)
[nan, nan, nan, ~2.0] = Plus(Tanh(Minus(Neg(y), x)), -1)
[nan, nan, nan, ~1.0] = Tanh(Minus(Neg(y), x))
[nan, nan, nan, 0.0] = Pow(y, Neg(x))
[nan, nan, nan, 1.0] = Exp(Pow(y, Neg(x)))
[nan, nan, nan, 2.0] = Plus(Exp(Pow(y, Neg(x))), 1)
[nan, nan, nan, inf] = Plus(y, x)
[nan, nan, ~inf, ~inf] = Neg(y)
[nan, nan, ~2.0, ~2.0] = Plus(Tanh(Neg(y)), -1)
[nan, nan, ~1.0, ~1.0] = Tanh(Neg(y))
[nan, nan, ~0.0, ~0.0] = Div(-1, y)
[nan, nan, 1.0, 1.0] = Exp(Div(-1, y))
[nan, nan, 2.0, 2.0] = Minus(1, Tanh(Neg(y)))
[nan, nan, inf, inf] = y
[nan, ~inf, nan, ~inf] = Neg(x)
[nan, ~inf, ~inf, ~inf] = Neg(Hypot(y, x))
[nan, ~2.0, nan, ~2.0] = Plus(Tanh(Neg(x)), -1)
[nan, ~2.0, ~2.0, ~2.0] = Plus(Tanh(Neg(Hypot(y, x))), -1)
[nan, ~1.0, nan, ~1.0] = Tanh(Neg(x))
[nan, ~1.0, ~1.0, ~1.0] = Tanh(Neg(Hypot(y, x)))
[nan, ~0.0, nan, ~0.0] = Div(-1, x)
[nan, ~0.0, ~0.0, ~0.0] = Div(-1, Hypot(y, x))
[nan, 1.0, nan, 1.0] = Exp(Div(-1, x))
[nan, 1.0, 1.0, 1.0] = Exp(Div(-1, Hypot(y, x)))
[nan, 2.0, nan, 2.0] = Minus(1, Tanh(Neg(x)))
[nan, 2.0, 2.0, 2.0] = Minus(1, Tanh(Neg(Hypot(y, x))))
[nan, inf, nan, inf] = x
[nan, inf, inf, inf] = Hypot(y, x)
[~inf, ~inf, ~inf, ~inf] = NegInf
[~2.0, ~2.0, ~2.0, ~2.0] = Plus(-1, -1)
[~1.0, ~1.0, ~1.0, ~1.0] = -1
[0.0, 0.0, 0.0, 0.0] = 0
[1.0, 1.0, 1.0, 1.0] = 1
[2.0, 2.0, 2.0, 2.0] = Plus(1, 1)
[inf, inf, inf, inf] = Inf


IEEE 754, 3.3
minimal encoding size:
b, radix, 2
p, precision, ?
emax, maximum exponent, ?
emin = always 1 - emax


I deduce:
p = t + 1
    (because you get one implied bit by normalization)
t = 1
% using the table
emax = $2^{k-p-1} - 1$, which is $2^{4-2-1}-1 = 2^1-1 = 1$
bias = emax
emin = 1 - emax = 0

I think there's still some ambiguity about whether we have
emax = 1, emin = 0,
or
emax = 0, emin = -1,
probably both work, but I think the former has more support
in the docs.

\comment{
a normal value is
(-1)^sign x b^{exp} x significand
(I think significand = mantissa?)

by standard, need:
- sign bit
- exponent could maybe be empty, though it's sort of implied
that you just have emax=0, emin= -1, so one bit
- m could probably be empty (no bits mantissa = 0)


DID YOU KNOW?
``every finite floating point numbe is an integral multiple
of the smallest subnormal magnitude b^{emin} * b^{1 - p}''

fewest exponent bits w?
The range of the encoding's biased exponent E shall include:
- every integer between 1 and 2^{w}−2, inclusive, to encode normal numbers
- the reserved value 0 to encode +/-0 and subnormal numbers
- the reserved value 2^{w}−1 to encode +/-inf and NaNs

so we need at least 0,
and a reserved value for inf & nan,
so if w = 1, we'd have 0 and 2^1-1 = 1,
but this seems to violate the spec because you need
``every integer between 1 and 2^w-2 inclusive'', and here 2^w-2 is 0,
so either this interval is inverted (arguable?) and that's why it
doesn't include its endpoints, or it's in violation
but with w = 2,

2^2 - 1 = 0b11 = 3 (inf, nan)
1 to 2^2 - 2 = 0b01, 0b10 = 1,2 (normals)
0 = 0b00 = 0 (zero, subnormals)

(Note this is the biased exponent E, not e)

2 bits looks clearly legal.

 s       w bits           t bits
[sign][ exponent ][  trailing significand T   ]

when e=0b11, t determines inf or nan.
- nan is any nonzero sign is ignored.
(recall, nans can carry info, so that is why they get
the nonzero case)
- inf is zero

so we clearly need 1 t-bit to encode inf vs nan,
but it seems that this is enough

so probable case is four bits:

 - sign
 - two exponent bits
 - one mantissa bit

The sign is always 0 (positive infinity; doesn't matter for nan)
and the exponent is always 11 (since we always encode nan or
positive inf). Unless we do any intermediate calculations
(could check what all the intermediate values are in the above
expressions and make sure they fit in 4 bits), a hardware
realization of this could just hard-wire 3 of 4 bits? Is that
better or worse?

so another version of this joke is that a NAND gate is a super
fast FPU for calculating inf - max(x + y, -inf) on four-bit
IEEE 754 where the top 3 MSBs are fixed (although be careful,
nan = 1 and inf = 0 in this representation. Is that maybe a
better choice for this presentation? it just dualizes some
gates)

}

\section{The binary4 representation}

IEEE 754 natively defines several bit widths for floating-point
values, such as the 32-bit binary32 (aka ``single-precision float'')
and 64-bit binary64 (aka ``double-precision float''). The
specification is parameterized to allow other bit widths; for example,
half-precision 16-bit floats are common in GPU code for machine
learning applications (XXX cite). Smaller floats sacrifice precision,
but require less space and allow faster calculations. For our purposes
in this paper, since we only need to represent the two numbers \nan\ and
\inf, we are interested in the smallest possible representation. 

% TODO: what if each bit in the float is itself represented as
% inf or NaN?

This section describes the binary4 representation, a four-bit floating
point number that is clearly allowed by the IEEE 754 standard.

The representation of any floating-point number has a single sign bit,
some number $w$ of exponent bits, and some number $t$ of mantissa
bits. For binary32, $w = 8$ and $t = 23$; and with the sign bit we
have $23 + 8 + 1 = 32$ bits as expected. We need at least a sign bit,
but what are the smallest permissible values of $w$ and $t$?

The most stringent constraint on $w$ comes in IEEE-754-2008~3.4, which
states
\begin{quote}
  The range of the encoding's biased exponent E shall include:
  \begin{enumerate}[label=---]
    \item every integer between 1 and $2^w - 2$, inclusive, to encode
      normal numbers
    \item the reserved value 0 to encode $\plusminus 0$ and subnormal
      numbers
    \item the reserved value $2^w - 1$ to encode $\plusminus \infty$
      and NaNs.
  \end{enumerate}
\end{quote}

$E$ is the binary number encoded by $w$. It must include at least the
two special values consisting of all zeroes and all ones (second and
third clause). A conservative reading of ``every integer between 1 and
$2^w - 2$'' seems to require that $1 \leq 2^w - 2$ (otherwise how
could the interval be inclusive of its endpoints?), which would imply
that $w$ is at least 2. (However, see Section~\ref{sec:binary3} for
the hypothesized case where $w=1$.)

The representation of \nan\ and \inf\ are distinguished by the value
of $t$ when $E$ is all ones. We certainly need to distinguish these,
so $t = 1$ is the minimal size.

We have one sign bit, two exponent bits, and one mantissa bit, for a
total of four. Since ``single precision'' is 32 bits, ``half
precision'' is 16, 4 bits is ``eighth precision.'' Given how nicely
all this works out, shouldn't there be an {\tt eighth} base type in
most modern programming languages and GPUs? Since there are so few
values representable, it would be practical for all the standard
operations to be done in constant time via table lookups. All 16
possible values are given in Figure~\ref{fig:binary4}.


\begin{figure*}[tp]
\begin{tabular}{|l|c|c|c|c|c|c|}
  \hline
  parameter & {\bf binary4} & binary16 & binary32 & binary64 & binary128 & binary$_k$ \\
  \hline
  $k$, storage in bits & {\bf 4} & 16 & 32 & 64 & 128 & multiple of 32 \\
  $p$, precision in bits & {\bf 2} & 11 & 24 & 53 & 113 & $k$ - round(4 * log$_2$(k)) + 13 \\
  $emax$, maximum exponent e & {\bf 1} & 15 & 127 & 1023 & 16383 & $2^{k-p-1}$ - 1 \\
  $bias$ = $E$ - $e$ & {\bf 1} & 15 & 127 & 1023 & 16383 & emax \\
  $sign bits$ & {\bf 1} & 1 & 1 & 1 & 1 & 1 \\
  $w$, exponent width & {\bf 2} & 5 & 8 & 11 & 15 & round(4 * log$_2$(k)) - 13 \\
  $t$, trailing significand width & {\bf 1} & 10 & 23 & 52 & 112 & k - w - 1 \\
  $k$, storage width & {\bf 4} & 16 & 32 & 64 & 128 & 1 + w + t \\
\hline
\end{tabular}
\caption{Parameters for the newly-introduced {\bf binary4} encoding
  for IEEE-754, compared to the standard widths (see Table~3.5 in the
  standard\cite{ieee754}). } \label{fig:binary4}
\end{figure*}

% bias = 1

\begin{figure}[h]
\begin{tabular}{|l@{\,}c@{\,}l|p{2.5in}|}
\hline
  $s$ & $E$ & $T$ & value \\
  \hline
0 & 00 & 0 &   $+0$ \\
0 & 00 & 1 &    subnormal: % $2^{emin} * 2^{1-p} * T$ =
                $2^0 * 2^{1-2} * 1$ =
                $1 * \sfrac{1}{2} * 1$ = 0.5 \\
0 & 01 & 0 &    normal: % $2^{E - bias} * (1 + 2^{1-p} * T)$
                $2^0 * (1 + \sfrac{1}{2} * 0)$ = 1 \\
0 & 01 & 1 &    $2^0 * (1 + \sfrac{1}{2} * 1)$ = 1.5 \\
0 & 10 & 0 &    $2^1 * (1 + \sfrac{1}{2} * 0)$ = 2 \\
0 & 10 & 1 &    $2^1 * (1 + \sfrac{1}{2} * 1)$ = 3 \\
0 & 11 & 0 &   $+\infty$ \\
0 & 11 & 1 &    \nan \\
1 & 00 & 0 &   $-0$ \\
1 & 00 & 1 &   $-0.5$ \\
1 & 01 & 0 &   $-1$ \\ 
1 & 01 & 1 &   $-1.5$ \\
1 & 10 & 0 &   $-2$ \\
1 & 10 & 1 &   $-3$ \\ 
1 & 11 & 0 &   $-\infty$ \\
1 & 11 & 1 &    \nan \\
\hline
\end{tabular}
\caption{All 16 values representable in binary4 floating-point.
  The format works reasonably well even at this very low precision,
  although note how many of the values are not finite.} \label{fig:allvalues4}
\end{figure}


\subsection{The hypothesized binary3 format} \label{sec:binary3}
The IEEE-754 representation clearly requires a sign bit, and for this
purpose we need at least one bit for the mantissa in order to
distinguish \nan\ and \inf. It is perhaps a stretch of the wording,
but arguably the spec permits a 1-bit exponent ($w = 1$). To
rationalize this we need to interpret the phrase ``every integer
between $1$ and $2^w-2$ inclusive'' (that is, between 1 and 0 inclusive)
as denoting the empty set.

With one bit for sign, exponent, and mantissa, we can represent just 8
different values. Here $emax$ is $0$, and the standard clearly
requires $emin = 1 - emax$, so $emin = 1$. Certainly fishy for $emin$
to be larger than $emax$, but we can just not stress out about it; the
representable values are all reasonable-looking
(Figure~\ref{fig:allvalues3}).

% here, precision is still 2 (just one mantissa bit)
% so emax is 2^{k-p-1}-1, which is 2^{3-2-1}-1 = 0
% bias = 0
% emin is 1 - emax = 1 (!?)
% this is kind of absurd, it may still "work"?

\begin{figure}[h]
\begin{tabular}{|l@{\,}c@{\,}l|p{2.5in}|}
\hline
  $s$ & $E$ & $T$ & value \\
  \hline
0 & 0 & 0 &   $+0$ \\
0 & 0 & 1 &    subnormal: % $2^{emin} * 2^{1-p} * T$ =
                $2^1 * 2^{1-2} * 1$ =
                $2 * \sfrac{1}{2} * 1$ = 1 \\
0 & 1 & 0 &   $+\infty$ \\
0 & 1 & 1 &    \nan \\
1 & 0 & 0 &   $-0$ \\
1 & 0 & 1 &   $-1$ \\
1 & 1 & 0 &   $-\infty$ \\
1 & 1 & 1 &    \nan \\
\hline
\end{tabular}
\caption{All 8 values of the hypothetical binary3 representation.
  There are no normal values; the only finite values are the
  positive and negative zero and a single subnormal which
  denotes $1$ (or $-1$).
} \label{fig:allvalues3}
\end{figure}

% TODO: We could do without a mantissa if we didn't care about
% distinguishing NaN and inf. With an empty t, we should
% consider it to have value 0 (true? maybe it is defined as
% a product so the empty string is the multiplicative unit 1?)
% which means that we encode +/- inf, and nan cannot be encoded.
% Now the operations will not be closed since e.g. 0/0 needs to
% result in nan.

% TODO: What if we have no mantissa or exponent, just a sign bit?
% When E=0 and T=0, we can only encode +/- zero. This definitely
% involves some leaps (and is not closed as above), but could
% we find expressions that work for just +/- 0?

% hardware notes:
% http://openocd.org/ looks like a good way to do in-device
% programming like over SWD, because it runs on the pi. A board
% could have jumpers or even software route to different
% gpio pins. (You would need to run at 3v3 for this of course.)
% I guess make sure the chip is compatible, but there are lots
% explicitly listed.

% Some supported devices:
% AT91SAM7 - no fp in the M7
% at91sam3u - no fp here either :/

% LPC1700, LPC1800,
% LPC2000, LPC4300, , STR7x, STR9x, LM3, STM32x and
% EFM32

% ok the STM32f3 is supported, has FPU, 51 IO pins,
% specifically like the
% https://www.mouser.com/ProductDetail/STMicroelectronics/STM32F303RDT7?qs=%2Fha2pyFaduikpRt0bOqlahEvgTujL2BrdYPen6RZRRk%3D

% here is an example breakout:
% http://ebrombaugh.studionebula.com/embedded/stm32f303breakout/index.html
% requires a crystal, voltage regulator, and some filter capacitors,
% but looks manageable. (And the chip has some 5V tolerant TTL I/O
% pins too, which could be nice for other projects)
% also maybe a ferrite bead (for input to analog VDDA, so probably
% not needed for this project)

Idea: Make a FPU. How small could it be?
I guess the simplest thing that could be described as faithful
would compute the NAND function of two binary3 numbers, but on
the entire domain. There are six input pins, three output pins,
and a table size of 8^2 = 64.

A nice simple nand function is
f(x, y) = inf - max(x + y, -inf)

What is a simple implementation of this function?
It can only output inf or nan, so that's good. Of the three
output pins, the first is always 0, the second always 1, so
at worst we would just need a table of 64 bits for this last
value.

Further simplification:

if isfinite(x) && isfinite(y)
then
  // inf - (x + y) = inf
  0
else
// x + y is nan when x is nan, y is nan, or x and y are
// infinites with different signs. If they are both -inf,
// then we have max(-inf, -inf) anyway, which is the same
// as max(nan, -inf). So we have
  if x == inf && y == inf
    // both positive infinities
    // inf - inf = nan
    1
  else
    // inf - -inf = inf
    0

So ultimately this function only returns 1 in the case
that both inputs are exactly +inf, the pattern 0 1 0.

If the inputs are a0 a1 a2, b0, b1, b2, and outputs
are c0, c1, c2, then:
c0 = 0
c1 = 1
c2 = !a0 && !a2 && !b0 && !b2 && a1 && b1

// Could adapt find.sml to find solutions for this,
// using only NAND. Only 2^6 = 64 possible values
// for these variables

let x = nand(!a0, !a2)
let y = nand(!b0, !b2)
let z = nand(a1, b1)


let w = Not(Nand(Not(x), Not(y)))
let out = Not(Nand(w, Not(z))

... this is pretty much a straightforward Nand construction.
I guess it doesn't really simplify much across units.

or with more gates,
Not(Or(z, Or(y, x)))

and I mean obviously it's just
And(Not (Or (a0, a2, b0, b2)), a1, b1)
if you have all the gates directly.

we can certainly make chips implementing each of these gates,
since we have floating point implementations of each, but
then we sort of lose the final joke since we'd just build
a floating point unit that computes the NAND instruction.

so let's try the pure-nand approach.

implementing the nots also with nands, this works out to 14
gates, which is ``just'' three chips. very reasonable. I was
actually hoping for it to be a little more ridiculous.
(maybe there is a lower pin-count device with an easier
package? programming them is still going to be a huge pain)

if we used the meta-nand gates, we'd have 14 * 3 = 42 chips,
which is OTOH probably too many.

with this meta-fpu, we have 2 binary3 inputs and 1 binary3
output, but each ``bit'' is represented as a binary3 number
itself. So this makes (3 * 2 + 1) * 3 = 27 wires to interface
with it. This is exactly as many GPIO the pi has, so that
is good, although IIRC a few pi pins have special annoyances,
and it doesn't leave us any space for a programmer.
Maybe just best to make a separate programmer and connect it
up to different test points manually. Then we could reuse
the programmer pretty easily.

Note that in a pinch we can ``cheat'' pretty easily here
since some of the pins have constant values. Probably justifiable
is that the meta-nand always outputs 0 (nan) for g and 1 (inf) for h,
plus two of the bits of i are fixed because it will only represent
nan or inf. But really would be ``prettiest'' to not make this cheat,
I mean, it's a pretty silly excursion anyway.

oh jk there is actually gpio0 too, so 28 pins on pi.

#  0 & 0 & 0 &   $+0$ \\
#  0 & 0 & 1 &   +1
#  0 & 1 & 0 &   $+\infty$ \\
#  0 & 1 & 1 &    \nan \\
#  1 & 0 & 0 &   $-0$ \\
#  1 & 0 & 1 &   $-1$ \\
#  1 & 1 & 0 &   $-\infty$ \\
#  1 & 1 & 1 &    \nan \\


% would be cool if the meta NAND gate were addressable
% from the rasberry pi (well duh of course it is?) because
% then we could also use it to simulate itself in software

\nocite{ieee754}
\bibliography{nand}{}
\bibliographystyle{plain}

\end{document}
