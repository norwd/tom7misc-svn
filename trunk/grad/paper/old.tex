% For posterity, old approaches to IsntZero
\begin{lstlisting}[language=C]
hfluint8 hfluint8::IsntZero(hfluint8 a) {
  // A simple way to do this is to extract all the (negated) bits
  // and multiply them together. But this would mean multiplying
  // a by itself, and so is not linear.
  //
  // Instead, do the same but ADD the bits. Now we have a
  // number in [0, 8]. So now we can do the whole thing again
  // and get a number in [0, 4], etc.

  half num_ones = (half)0.0f;
  hfluint8 aa = a;
  for (int bit_idx = 0; bit_idx < 8; bit_idx++) {
    hfluint8 aashift = RightShift1(aa);
    half a_bit = aa.h - LeftShift1Under128(aashift).h;
    num_ones += a_bit;
    aa = aashift;
  }

  ECHECK(num_ones >= (half)0.0f && num_ones <= (half)8.0f);

  // now count the ones in num_ones.
  aa = hfluint8(num_ones);
  num_ones = (half)0.0f;
  for (int bit_idx = 0; bit_idx < 4; bit_idx++) {
    hfluint8 aashift = RightShift1(aa);
    half a_bit = aa.h - LeftShift1Under128(aashift).h;
    num_ones += a_bit;
    aa = aashift;
  }

  ECHECK(num_ones >= (half)0.0f && num_ones <= (half)4.0f);

  // and again ...
  aa = hfluint8(num_ones);
  num_ones = (half)0.0f;
  for (int bit_idx = 0; bit_idx < 3; bit_idx++) {
    hfluint8 aashift = RightShift1(aa);
    half a_bit = aa.h - LeftShift1Under128(aashift).h;
    num_ones += a_bit;
    aa = aashift;
  }

  ECHECK(num_ones >= (half)0.0f && num_ones <= (half)3.0f);

  // and again ...
  aa = hfluint8(num_ones);
  num_ones = (half)0.0f;
  for (int bit_idx = 0; bit_idx < 2; bit_idx++) {
    hfluint8 aashift = RightShift1(aa);
    half a_bit = aa.h - LeftShift1Under128(aashift).h;
    num_ones += a_bit;
    aa = aashift;
  }

  ECHECK(num_ones >= (half)0.0f && num_ones <= (half)2.0f);

  // Now num_ones is either 0, 1, or 2. Since 1 and 2 is each
  // represented with a single bit, we can collapse them with
  // a shift and add:
  //   num_ones    output
  //         00         0
  //         01         1
  //         10         1
  hfluint8 nn(num_ones);
  return hfluint8(nn.h + RightShift1(nn).h);
}

static hfluint8 OldIsZero1(hfluint8 a) {
  // We know IsZero returns 1 or 0.
  // return hfluint8(1.0_h - IsntZero(a).h);

  hfluint8 aa = a;
  // true if everything is 1.0 so far
  half res = 1.0_h;
  for (int bit_idx = 0; bit_idx < 8; bit_idx++) {
    // leftmost bit
    half bit = RightShift<7>(aa).h;
    half nbit = (1.0_h - bit);
    // res = res & ~bit
    res = RightShiftHalf1(nbit + res);
    // aa = LeftShift<1>(aa);
    // We already have the high bit, so mask it off if necessary
    aa = hfluint8(aa.h - bit * 128.0_h);
    aa = LeftShift1Under128(aa);
  }

  return hfluint8(res);
}
\end{lstlisting}


% old approach to If

\begin{lstlisting}[language=C]
  // For cc = 0x01 or 0x00 (only), returns c ? t : 0.
hfluint8 hfluint8::If(hfluint8 cc, hfluint8 t) {
  // Multiplying cc * t gives us what we want, but this may
  // violate linearity (e.g.~if the two inputs are the same).

  // Could do this by spreading the cc to 0xFF or 0x00 (cc * 255 will
  // do that) and using AND, but it is faster to just keep consulting
  // the ones place of cc.

  half kept = GetHalf(0x0000);
  hfluint8 tt = t;
  for (int bit_idx = 0; bit_idx < 8; bit_idx++) {
    // Low order bit as a - ((a >> 1) << 1)
    hfluint8 ttshift = RightShift1(tt);
    half t_bit = tt.h - LeftShift1Under128(ttshift).h;
    // Computes 2^bit_idx
    const half scale = GetHalf(0x3c00 + 0x400 * bit_idx);

    const half and_bits = RightShiftHalf1(t_bit + cc.h);
    kept += scale * and_bits;

    // and keep shifting down
    tt = ttshift;
  }

  return hfluint8(kept);
}
\end{lstlisting}
