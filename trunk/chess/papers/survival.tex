
\begin{document}

Survival in Chessland

If you are forced to play chess to the death, you are in trouble,
because most people are not good at chess (for example, the author)
and yet want to live.\footnote{It is easy for two players to
  collaborate to produce a draw, especially by simply agreeing to a
  draw at the outset of the game (if allowed). Some tournament formats
  forbid the players from agreeing to a draw verbally before a certain
  point (e.g. 30 moves). There are always other routes to a draw, for
  example by stalemate or repeating the same position three times.
  Collaboratively producing such situations is easy, but this strategy
  is not likely a stable equilibrium: Players can often gain a
  substantial advantage by going ``off script'' and instead trying to
  win the game. Additionally, sometimes the terms of
  chess-to-the-death do not allow the players to communicate at all
  beforehand, nor during the game. If this is the case, then it may be
  difficult to agree on the approach to drawing, let alone establish
  that this is both player's desire. Since the rules of
  chess-to-the-death can't forbid us from colluding at this moment, I
  hereby declare that the following is the correct approach: 
%
  1. Nf3. This is a reasonable opening move for white (``Zukertort opening'')
  which can transpose into several common systems (e.g. King's
  Indian). Since the knight can move back to g1 on the next move,
  knight moves are the fastest route to a draw by repetition. This
  has a good chance of signaling to a wise player that a draw is
  desired. The player should make this move after pondering carefully
  for some time, and then looking meaningfully into the other player's
  eyes. 
  1. ... Nf6. This is both a strong response for black in a real game,
  and simultaneously a signal that a draw is desired. The other
  advantage is that very weak players may simply copy what white does.
  In doing so, they will also play this move. 
  2. Ng1?!. This is a terrible move for white, but clearly signals
  the intention to draw.
  2. ... Ng8!. ``Fool's Draw Accepted.'' The starting position is reached
  for a second time.
  3. Nf3 Nf6. At this point the game should clearly continue repeating
  the sequence, although since we are in the start position, white has
  any number of strong opening moves available. Signaling the draw line
  and then 3. d4!? may be pyschologically devastating.
  4. Ng1 Ng8 1/2-1/2. The starting position is reached for the third
  time, which by rule is a draw.

  Since this is one of the shortest possible routes to a draw, I hereby
  dub this line the ``Fool's Draw,'' by analogy with the Fool's Mate.

  If 1. ... Nc6 or another Knight's move, white can also consider
  continuing in the obvious way. However after 1. ... d5, black
  has refused or not noticed the draw. Fortunately, white is still
  in a good position to play the game normally. White can try
  to be more obvious with 2. Ng1, but if black is choosing to just
  play normally, white takes a distinct handicap by doing so.

  The biggest risk for white is that black does not play 2. ... Ng8
  but rather a normal move like 2. ... g6 (``Fool's Draw Betrayed'').
  This can happen if black is not metagaming at all (for Nf6 is a
  normal response to the normal Nf3), or if black is an exceptionally
  shrewd metagamer (tricking white into wasting two tempos with Ng1
  by pretending to be cooperating).

  Of course, this all relies on the assumption that if
  chess-to-the-death ends in a draw, the players are spared or allowed
  to repeat indefinitely. If both players are actually executed, then
  this line is truly a Fool's Draw! }
%
But what if you are forced to {\it be one of the chess pieces} to the
death? That is, your little soul inhabits one of the 32 pieces or
pawns and your soul is vanquished if that piece is eliminated. Now it
doesn't matter whether you're good or bad at chess, but

In formal chess, the king can never be captured: The game ends when
the king is attacked but cannot move, and it is illegal to make a move
that leaves the king attacked. The king's death is implied, of course,
but it is seen as more poetic to end the game prior to this point.

For the sake of this question, we'll consider the the white king to
``die'' if white loses (i.e., is checkmated), and likewise for black.
Otherwise, of course, the best chances of survival would trivially be
with the two kings, since they always survive. Loss includes
resignation, since most high-level games actually end once the
defeated player agrees that loss is inevitable. We can think of this
common case like king seppuku. Many games also end in time forfeit,
which is like the king's poor diet and lifestyle choices leading to
a death by natural causes.

Neither side has a decisive advantage, and many games end in a draw,
with both kings surviving. So the survival chances of a king are
clearly greater than 50\%---pretty decent odds. Is it possible that
any other piece has even better chances?


Here are my guesses:

\begin{itemize}
\item Black and white are probably not significantly different. That
  is, the a1 and a8 rooks probably have about the same survival
  chances (it's known that white has a slight statistical advantage
  but it is probably only around 1\%). So these guesses will be
  written about white's pieces.
\item The d2 and e2 pawns are very active in common openings, and
  are frequently captured as part of those openings. I think they
  are the least likely to survive overall.
\item Bishops and knights are often involved in the opening and
  midgame, and often exchanged nonchalantly. I think they all have
  relatively low survival chances.
\item Although the queen is very valuable, a queen exchange is often
  forced for games that enter the endgame.
\item Rooks tend to be late-game pieces, because they are difficult
  to get out of their corners (and at most one can be activated
  by the fastest method, castling) and are relatively valuable.
\item This leaves the non-central pawns. These are the hardest to
  predict, and they are hard to think about (at least for me) because
  when e.g. the a2 pawn recaptures the b3 pawn that it supported, I
  just think of this as the b3 pawn. Of these pawns, b2 and g2 are
  somewhat weak because they are undefended once the bishop is
  developed (cf. the famous ``poison pawn'' at b2). On the other hand,
  in the fianchetto configuration, this pawn is very strong and often
  survives the entire game without leaving the third rank. Since pawn
  chains usually progress towards the middle of the board, the a2 pawn
  is more likely to be supporting than supported. This both leaves it
  weak to capture, but prone to recapturing. Outside pawns block one's
  own rook, although for this same reason they often clear the file by
  capturing (and so survive). They are also commonly used to push into
  a well-defended king's territory (e.g. in the fianchetto); kingside
  castling is more common, so this means that the h pawns are often
  lost to this fate.
\end{itemize}

This leaves my final ranking, from least surviving to most surviving,
as:

16. d2,
15. e2,
14. b1 knight,
13. g1 knight,
12. c1 bishop,
11. f1 bishop,
10. queen,
9. king,
8. a1 rook,
7. h1 rook,
6. b2,
5. h2
4. a2
3. g2
2. c2
1. f2

% Ben, David, Jim also wagered guesses

% Battle chess
% Feminist game 

\end{document}
