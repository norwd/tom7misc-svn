
\begin{document}

Survival in Chessland


In formal chess, the king can never be captured: The game ends when
the king is attacked but cannot move, and it is illegal to make a move
that leaves the king attacked. The king's death is implied, of course,
but it is seen as more poetic to end the game prior to this point.

For the sake of this question, we'll consider the the white king to
``die'' if white loses (i.e., is checkmated), and likewise for black.
Otherwise, of course, the best chances of survival would trivially be
with the two kings, since they always survive. Loss includes
resignation, since most high-level games actually end once the
defeated player agrees that loss is inevitable. We can think of this
common case like king seppuku. Many games also end in time forfeit,
which is like the king's poor diet and lifestyle choices leading to
a death by natural causes.

Neither side has a decisive advantage, and many game end in a draw,
with both kings surviving. So the survival chances of a king are
clearly greater than 50\%---pretty decent odds. Is it possible that
any other piece has even better chances?


Here are my guesses:

\begin{itemize}
\item Black and white are probably not significantly different. That
  is, the a1 and a8 rooks probably have about the same survival
  chances (it's known that white has a slight statistical advantage
  but it is probably only around 1\%). So these guesses will be
  written about white's pieces.
\item The d2 and e2 pawns are very active in common openings, and
  are frequently captured as part of those openings. I think they
  are the least likely to survive overall.
\item Bishops and knights are often involved in the opening and
  midgame, and often exchanged nonchalantly. I think they all have
  relatively low survival chances.
\item Although the queen is very valuable, a queen exchange is often
  forced for games that enter the endgame.
\item Rooks tend to be late-game pieces, because they are difficult
  to get out of their corners (and at most one can be activated
  by the fastest method, castling) and are relatively valuable.
\item This leaves the non-central pawns. These are the hardest to
  predict, and they are hard to think about (at least for me) because
  when e.g. the a2 pawn recaptures the b3 pawn that it supported, I
  just think of this as the b3 pawn. Of these pawns, b2 and g2 are
  somewhat weak because they are undefended once the bishop is
  developed (cf. the famous ``poison pawn'' at b2). On the other hand,
  in the fianchetto configuration, this pawn is very strong and often
  survives the entire game without leaving the third rank. Since pawn
  chains usually progress towards the middle of the board, the a2 pawn
  is more likely to be supporting than supported. This both leaves it
  weak to capture, but prone to recapturing. Outside pawns block one's
  own rook, although for this same reason they often clear the file by
  capturing (and so survive). They are also commonly used to push into
  a well-defended king's territory (e.g. in the fianchetto); kingside
  castling is more common, so this means that the h pawns are often
  lost to this fate.
\end{itemize}

This leaves my final ranking, from least surviving to most surviving,
as:

16. d2,
15. e2,
14. b1 knight,
13. g1 knight,
12. c1 bishop,
11. f1 bishop,
10. queen,
9. king,
8. a1 rook,
7. h1 rook,
6. b2,
5. h2
4. a2
3. g2
2. c2
1. f2

% Ben, David, Jim also wagered guesses

\end{document}
