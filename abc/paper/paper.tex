\documentclass[twocolumn]{article}
\usepackage[top=1.1in, left=0.85in, right=0.85in]{geometry}

\usepackage{supertabular}
\usepackage{amsthm}
\usepackage{relsize}
\usepackage{amsmath}
\usepackage{amssymb}
% \usepackage{code}
\usepackage{graphicx}
\usepackage{fancyvrb}
\usepackage{url}
\usepackage{textcomp}

\pagestyle{empty}

\newcommand\comment[1]{}

\newcommand\st{$^{\mathrm{st}}$}
\newcommand\nd{$^{\mathrm{nd}}$}
\newcommand\rd{$^{\mathrm{rd}}$}
\renewcommand\th{$^{\mathrm{th}}$}
\newcommand\tm{$^{\mbox{\tiny \textsc{tm}}}$}

% nice fractions
\newcommand\sfrac[2]{{}\,$^{#1}$\!/{}\!$_{#2}$}

\newcommand\citef[1]{\addtocounter{footnote}{1}\footnotetext{\cite{#1}}\ensuremath{^{\mbox{\footnotesize [\thefootnote]}}}}

\usepackage{ulem}
% go back to italics for emphasis, though
\normalem

\begin{document} 

\title{This is the Title}
\author{Dr.~Tom~Murphy~VII~Ph.D.\thanks{
    Copyright \copyright\ 2017 the Regents of the Wikiplia Foundation.
    Appears in SIGBOVIK 2017 with the utmost concern of the
    Association for Computational Heresy; {\em IEEEEEE!} press,
    Verlag--Verlag volume no.~0x2017. BMD 0.00} }

\renewcommand\>{$>$}
\newcommand\<{$<$}

\date{1 April 2017}

\maketitle

\section*{Abstract}
This is the abstract

\vspace{1em}
{\noindent \small {\bf Keywords}:
  keywords
}

\section{Introduction}

% texticutable

% ``characters'' ``words'' and ``paragraphs'' and ``pages''

% Literate executable, knuth
% In the dead space of the binary, put this paper or part of it

% ``PISC'' = Printable Instruction Set Computer, etc.

It's a common experience that each time you program a solution to a
hard problem, you do it a different way, due to a lack of complete
satisfaction with the way you did it last time. True to form, the
ABC compiler exercises this technique *within the same compiler*,
using a different approach to represent each of the intermediate
languages.

TODO: Convert byte table thing into ascii. Unfortunately can't fit
the whole 256x256 thing, so maybe 128x128 with a small amount of
cropping, then a 1:1 zoom on the sierpinski triangle part?

Another possibility is to include readable debug information (e.g.
frame layouts) in a separate section.

\bibliographystyle{IEEEtran}
% \bibliographycomment{} % nothing
\bibliography{paper}

\end{document}
