\documentclass[twocolumn]{article}
\usepackage[top=1.1in, left=0.85in, right=0.85in]{geometry}

\usepackage{supertabular}
\usepackage{amsthm}
\usepackage{relsize}
\usepackage{amsmath}
\usepackage{amssymb}
% \usepackage{code}
\usepackage{graphicx}
\usepackage{fancyvrb}
\usepackage{url}
\usepackage{textcomp}

\pagestyle{empty}

\newcommand\comment[1]{}

\newcommand\st{$^{\mathrm{st}}$}
\newcommand\nd{$^{\mathrm{nd}}$}
\newcommand\rd{$^{\mathrm{rd}}$}
\renewcommand\th{$^{\mathrm{th}}$}
\newcommand\tm{$^{\mbox{\tiny \textsc{tm}}}$}

% nice fractions
\newcommand\sfrac[2]{{}\,$^{#1}$\!/{}\!$_{#2}$}

\newcommand\citef[1]{\addtocounter{footnote}{1}\footnotetext{\cite{#1}}\ensuremath{^{\mbox{\footnotesize [\thefootnote]}}}}

\usepackage{ulem}
% go back to italics for emphasis, though
\normalem

\begin{document} 

\title{This is the Title}
\author{Dr.~Tom~Murphy~VII~Ph.D.\thanks{
    Copyright \copyright\ 2017 the Regents of the Wikiplia Foundation.
    Appears in SIGBOVIK 2017 with the utmost concern of the
    Association for Computational Heresy; {\em IEEEEEE!} press,
    Verlag--Verlag volume no.~0x2017. BMD 0.00} }

\renewcommand\>{$>$}
\newcommand\<{$<$}

\date{1 April 2017}

\maketitle

\section*{Abstract}
This is the abstract

\vspace{1em}
{\noindent \small {\bf Keywords}:
  keywords
}

\section{Introduction}

% texticutable

% ``characters'' ``words'' and ``paragraphs'' and ``pages''

% Literate executable, knuth

% ``PISC'' = Printable Instruction Set Computer, etc.

It's a common experience that each time you program a solution to a
hard problem, you do it a different way, due to a lack of complete
satisfaction with the way you did it last time. True to form, the
ABC compiler exercises this technique *within the same compiler*,
using a different approach to represent each of the intermediate
languages.

Another possibility is to include readable debug information (e.g.
frame layouts) in a separate section.

histogram of actual bytes used in code segment?

interesting bugs:
  DI <- [DI]

words contained in this paper

opportunity for style points, like making your phone number into
a word:

 j ? X 4 ?   (4 can alternatively be $ or ,) 

 TOP PERFORMANCE IMPROVEMENTS:
  - don't jump at end of every block
  - don't assume a block needs to be split just because it's 0x7e-0x20.
    measure from the last jump!

The CISC Ridiculous

awesome text art headings?

Truly self-documenting
For good reasons that I will explain later...

I was kidding when I said the x86 architecture is elegant.

Dr. Dobbs Journal typing in the program

ES segment must be used for IN?

TODO: X86 vs x86
TODO: make sure to ctrl-u meta-q every paragraph

(Idea: ladder is actually interspersed in code, by putting some long instruction
byte (that takes several arguments) before DEC SP JNZ +7e? We can jump directly
to the jump, but if executed in a normal synchronized flow, then it executes some
instruction that does nothing (load immediate?).)

RISC ROLL

songs to do: https://www.youtube.com/watch?v=EJkJbUbvYJA
 - mario theme.

In video: show error in intel manual http:\\www.intel.com

video: bingo of tom 7 video themes (``retro'', ``alphabet,''
``fractal geometry'')

could write about optimizing the value table

Ideas for improvement before paper's submitted:
 - Implement 'void' return type
 - Make int literals actually 16 bit for now?
 - Implement multiplication and division
 - Implement switch
 - in paper.c, support multichannel sound.
 - implement sbrk primitive (just returns 0x8000?)
 
\bibliographystyle{IEEEtran}
% \bibliographycomment{} % nothing
\bibliography{paper}

\end{document}
