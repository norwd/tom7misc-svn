\documentclass[twocolumn]{article}
\usepackage[top=1.1in, left=0.85in, right=0.85in]{geometry}

\usepackage{supertabular}
\usepackage{amsthm}
\usepackage{relsize}
\usepackage{amsmath}
\usepackage{amssymb}
% \usepackage{code}
\usepackage{graphicx}
\usepackage{fancyvrb}
\usepackage{url}
\usepackage{textcomp}

\pagestyle{empty}

\newcommand\comment[1]{}

\newcommand\st{$^{\mathrm{st}}$}
\newcommand\nd{$^{\mathrm{nd}}$}
\newcommand\rd{$^{\mathrm{rd}}$}
\renewcommand\th{$^{\mathrm{th}}$}
\newcommand\tm{$^{\mbox{\tiny \textsc{tm}}}$}

% nice fractions
\newcommand\sfrac[2]{{}\,$^{#1}$\!/{}\!$_{#2}$}

\newcommand\citef[1]{\addtocounter{footnote}{1}\footnotetext{\cite{#1}}\ensuremath{^{\mbox{\footnotesize [\thefootnote]}}}}

\usepackage{ulem}
% go back to italics for emphasis, though
\normalem

\begin{document} 

\title{The glEnd() of Zelda}
\author{Dr.~Tom~Murphy~VII~Ph.D.\thanks{
    Copyright \copyright\ 2016 the Regents of the Wikiplia Foundation.
    Appears in SIGBOVIK 2016 with the danger of going alone of the
    Association for Computational Heresy; {\em IEEEEEE!} press,
    Verlag--Verlag volume no.~0x2016. 255 Rupees} }

\renewcommand\>{$>$}
\newcommand\<{$<$}

\date{1 April 2016}

\maketitle

\section*{Abstract}
3D ZELDA

\vspace{1em}
{\noindent \small {\bf Keywords}:
  small keys, boss keys, dungeon keys
}

\section{Introduction}

{\bf 1986. Hyrule.}\quad The Legend of {\it frickin'}~Zelda for the
Nintendo {\it freakin'}~Entertainment System. Need I say more? A {\it
  god damn} \uline{gold cartridge}. Fortunes made just from melting
the gold cartridge down to make gold teeth grilles, after carefully
extracting the even more precious ROM inside. A die-cut hole in the
box so that you could get a peek of the cartridge and presage that you
were getting some solid gold. A die cut little window that you could
palpate through the wrapping paper on Christmas~Eve, presaging some
epic thumb blisters in store for the coming weeks. Koji {\it freckin'}
Kondo. Koji Kondo whipping up a nice 8-bit arrangement of Bol\'ero as
the theme music until realizing at the last minute that this music was
copyrighted\footnote{Perhaps ironic, since Ravel's {Bol\'ero} itself
  was composed as a result of Ravel getting
  cop-blocked.\cite{wikipedia2016bolero} And surely some Zelda
  knock-off since then has included a Muzak ersatz of the Zelda theme!
  What is the longest documented sequence of compositions due to
  Copyright restrictions?} and so instead composing its epic theme in
{\it one day}??

\begin{figure}[ht]
\begin{center}
\includegraphics[width=\linewidth]{link3d}
\end{center}\vspace{-0.1in}
\caption{Technical diagram of mathematical equations.} \label{fig:link3d}
% \smallskip
% caption can go here...
% } 
\end{figure}

A gold cartridge that contained ROMs and a little swallowable battery
to keep the onboard SRAM powered up so that it could retain your epic
save game. A battery designed to last 70~years. Nothing could cause
you to lose your save game, even once you were half way through the
Second Quest. Unless your little brother starts a completely new game
and saves over your slot, earning him one of the most righteously
deserved clobberings this side of Inigo Montoya. Saves right on top of
your slot with a player called just \verb+A +. Right over your slot,
erasing it, and hasn't even picked up the {\it sword} yet. All you
have to do is step on the black square in the first room and the guy
gives you the sword and then dies. Jeez pooleaze. Saves right over
yours with no sword even though there are TWO OTHER UNUSED SLOTS and
you even wisely put your game in the third slot {\it exactly to avoid
  this kind of calamity}.

\medskip
Need I say more? Shall I say it? Apparently so. KIDS THESE DAYS. Kids
these days and their three dimensions. You can certainly make a
respectable case that the Zenith of the series was {\it Legend of
  Zelda III: A Link To The Past}, or even {\it Legend of Game Boy
  Zelda: I forgot What It's Called}. But there seems to be an
established consensus that {\it Legend Of Zelda LXIV: The Ocarina of
  Time} is the true masterpiece; that while the NES Original indeed
invented the genre, the technical limitations of the system
handicapped it. You're talking to me about technical {\it frockin'}
limitations? Like somehow appreciating the subtle wonder of the
universe and its mechanics requires me to look at tree people with
tetrahedra clubs for hands. And they say this of the 64\th\ game in
the series? Please jouize.

A fine game, don't get me wrong. But was its theme song composed in
one day? Did it come in a gold cartridge?

And the thing is: The original Zelda was built with enough forethought
that it {\em can} be rendered in first-person 3D. So here I will make
the game 3D so that it can be enjoyed by these kids and they can get
off my lawn, tapdancing around playing their titular Ocarinas. I will
make it possible to play the game in glorious HD, or even 4K
resolution, and then some (Figure~\ref{fig:link3d}). And while
we're at it, why not try to do it in a way that can play loads of NES
games in three {\it fr\"ockin'} dimensions? What's the worst that
could happen?


\section{Technical limitations}

As presaged, the NES does have certain technical limitations, which I
prefer to think of as constraints. Some basic details are important
for understanding how this works.

The NES has a modest CPU that executes game logic. It's a little 8-bit
baby puppy with access to 2 kilobytes of RAM. If you think about it,
the NES's 256x240 pixel screen is already 61,440 pixels, which if it
used the entirety of RAM to represent it, we'd have only
$0.2\overline{6}$ bits of RAM per pixel, which makes no sense. At XXX
MhZ, and a generous 1 instruction per pixel, we'd only be able to render
XXX frames per second! This doesn't make sense, and this is because
the CPU does not render the graphics on a NES.

The NES also contains a custom Picture Processing Unit---PPU---which
outputs 60 frames of NTSC or PAL video per second. In fact, since the
PPU also drives a CPU interrupt for each scanline and vblank (XXX
true?), and most games use this for timing,%
%
\footnote{Fun fact: NTSC, used in the Americas, has a ~60Hz frame
  rate. But PAL, used in Europe and Asia, has a 50Hz refresh rate.
  This means that many games run 16\% slower in Europe than the US, so
  if you grew up playing there, you were playing on {\em Easy} mode.
  {\tt\ ;-)}}
%
in many ways the CPU is subservient to the PPU, kind of like the PPU
is the {\it real} heart of the NES. In the context of this paper, you
could say that the CPU is the Ocarina of Time and the PPU is the
Legend of Zelda. It's just an analogy. The PPU is very complicated and
has tetrahedra for hands, like clubs. But the CPU communicates with
the PPU not by drawing pixels, but by giving a high-level description
of the screen (by writing it into PPU memory). The chief insight of
this work is that the PPU representation can be interpreted and
rendered in 3D instead of 2D. This allows us to use the exact original
game logic and only change the way it's viewed. Sometimes this even
allows us to see aspects of the game that are actually present but
were not visible on NES hardware. Often it makes the game more
difficult by making things invisible (enemies behind the player or
behind walls) or inducing motion sickness.

\subsubsection{Background}
The PPU has two major drawing facilities: Background and Sprites. Both
are made of ``tiles'', which are 8x8 graphics that are
usually\footnote{Some games trick the NES hardware by remapping tile
memory (e.g. in response to a memory read), even during a scanline. I
do not allow for such chicanery in this work.} stored in ROM. There
are two tile sets, each with 256 tiles. Figure~\ref{fig:ppu} shows the
tile sets for the starting location in the Zelda overworld.

\begin{figure}[ht]
\begin{center}
\includegraphics[width=\linewidth]{ppu1} \\
\vspace{1em}
\includegraphics[width=\linewidth]{ppu2}
\end{center}\vspace{-0.1in}
\caption{Pattern tables for the Zelda overworld. In Zelda, the top tiles
are used for sprites, and the bottom are used for the background graphics.
All tiles are 8x8 and use 4 color values (color 0 is transparent). These
color values index into 4 sprite palettes and 4 background palettes, with
some limitations.
} \label{fig:ppu}
\end{figure}

The NES background is described by the ``nametable'', which is an
array of 32x30 bytes giving the tile numbers to fill the screen with.
It's two rows shy of 256 probably for some reason having to do with
the aspect ratio of TVs or number of actual NTSC scanlines, but also
to make room for 64 bytes of attribute data. There are just two bits
per four tiles. These two bits select which of the four palettes to
use for each 2x2 group of tiles on the screen. It's interesting to
look for the tricks that artists use to work around the constraint on
the number of background colors. % XXX figure showing this off?

Because of this array of tiles, NES games have to be built of
high-level, block-like structures. In Zelda, this structure is
particularly (but not unusually) regular. Since each 2x2 tile group
(16x16 pixels) shares the same palette, the screens are built around
16x16 pixel grid-aligned blocks.

\subsubsection{Sprites}

Sprites, named in honor of the forest pixies whose homes were razed in
order to make space inside the PPU for them, are more complicated.

There are 64 sprites, which are all turned on or off together. Each
sprite is described by four bytes: Its $x$ and $y$ coordinates (these
can be placed at arbitrary integer pixel locations), the index of
the tile to draw, and an attribute byte:

\begin{tabular}{rrl}
    & bit \# & bit deets \\
\hline \\
MSB & 7      & Vertical flip \\
    & 6      & Horizontal flip \\
    & 5      & something weird \\
    & 4      & I don't know \\
    & 3      & unused? who can say \\
    & 2      & bit 2; could be anything \\
LSB & 1,0    & Palette index \\
\end{tabular}

Additionally, there are some global sprite settings. But 

Sprite Zero has some special properties. I don't want to tell you
about the properties, I just wanted to make a soda joke.

\nocite{murphy2013first}

\bibliographystyle{IEEEtran}
% \bibliographycomment{} % nothing
\bibliography{paper}

\end{document}
