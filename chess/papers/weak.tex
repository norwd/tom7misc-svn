
% [computermodern], 
\documentclass[numbers]{sigplanconf}

\usepackage{url}
% \usepackage{code}
% \usepackage{cite}
\usepackage{amsmath}
\usepackage{amssymb}
\usepackage{graphicx}
\usepackage{chessboard}

% \usepackage{chessfs}
\usepackage{adjustbox}

% Define black versions of pieces for inline use. Gross, but it works.
\newcommand{\Pawn}[1][1.3ex]{%
\adjustbox{Trim=4.3pt 2.6pt 4.3pt 0pt,width=#1,margin=0.2ex 0ex 0.2ex 0ex}{\BlackPawnOnWhite}%
}%
\newcommand{\Rook}[1][1.58ex]{%
\adjustbox{Trim=3.2pt 2.2pt 3.2pt 0pt,width=#1,raise=0ex,margin=0.1ex 0ex 0.1ex 0ex}{\BlackRookOnWhite}%
}%
\newcommand{\Knight}[1][1.85ex]{%
\adjustbox{Trim=2.3pt 2.35pt 2.5pt 0pt,width=#1,raise=-0.03ex,margin=0.14ex 0ex 0.14ex 0ex}{\BlackKnightOnWhite}%
}%
\newcommand{\Bishop}[1][1.79ex]{%
\adjustbox{Trim=2.3pt 2pt 2.3pt 0pt,width=#1,raise=-0.12ex,margin=0.1ex 0ex 0.1ex 0ex}{\BlackBishopOnWhite}%
}%
\newcommand{\Queen}[1][2.05ex]{%
\adjustbox{Trim=1.2pt 2.2pt 1.2pt 0pt,width=#1,raise=-0.08ex,margin=0.1ex 0ex 0.1ex 0ex}{\BlackQueenOnWhite}%
}%
\newcommand{\King}[1][1.95ex]{%
\adjustbox{Trim=2pt 2pt 2pt 0pt,width=#1,raise=-0.06ex,margin=0.13ex 0ex 0.13ex 0ex}{\BlackKingOnWhite}%
}%

\interfootnotelinepenalty=0

% lets me explicitly set a. or 1. etc. as enum label
\usepackage{enumitem}

\pagestyle{empty}

\usepackage{ulem}
% go back to italics for emphasis, though
\normalem

\usepackage{natbib}

\setlength{\footnotesep}{2em}

% \newcommand\comment[1]{}
\newcommand\sfrac[2]{\!{}\,^{#1}\!/{}\!_{#2}}

\begin{document} 

\conferenceinfo{SIGBOVIK~2019}{April 1, 2019, Pittsburgh, Pennsylvania, USA}
\copyrightyear{2019}
\title{Elo World,}
\subtitle{a framework for benchmarking weak chess engines}
\authorinfo{Dr. Tom Murphy VII Ph.D.}{tom7.org Foundation}{tom7@tom7.org}
\toappear{
Copyright \copyright\ 2019 the Regents of the Wikiplia Foundation.
Appears in SIGBOVIK 19119 with the
%
insufficient material 
%
of the Association for Computational Heresy; {\em IEEEEEE!}
press, Verlag-Verlag volume no.~0x40-2A. 1600 rating points}

\ccsdesc[500]{Evaluation methodologies~Tournaments}
\ccsdesc[500]{Chess~Being bad at it}
% \ccsdesc[100]{Software and its engineering~Control structures}
% \ccsdesc[300]{Theory of computation~Control primitives}

\terms
pawn, horse, bishop, castle, queen, king
\keywords
chess, bad

\setchessboard{showmover=false}

\newcommand\checkmate{\hspace{-.05em}\raisebox{.4ex}{\tiny\bf ++}}

\renewcommand\th{\ensuremath{{}^{\textrm{th}}}}
\newcommand\st{\ensuremath{{}^{\textrm{st}}}}
\newcommand\rd{\ensuremath{{}^{\textrm{rd}}}}
\newcommand\nd{\ensuremath{{}^{\textrm{nd}}}}
\newcommand\at{\ensuremath{\scriptstyle @}}

\date{1 April 2019}

\maketitle \thispagestyle{empty}

Fiddly bits aside, it is a solved problem to maintain a numeric rating
of players for some game (for example chess, but also sports,
e-sports, probably also z-sports if that's a thing). Though it has
some competition (suggesting the need for a meta-rating system to
compare them), the Elo Rating System~\cite{elo1978rating} is a simple
and effective way to do it.

The gist of this system is to track a score for each player. The
scores are defined such that the expected outcomes of games can
be computed from the scores (for example, a player with a rating
of 2400 should win 9/10 of her games against a player with a
rating of 2000). If the true outcome doesn't match the 

.. so we can just leave it at that


% Names:
% "Stalemate"
% "Weaksauce"
% "back rank"
% "blunder"
% something with ..elo..   "elo world"
% inFIDElity
% blunderbuss
% 

Interpolation, like Scoville scale.

% Dramatic failure of the "suicide king" strategy against
% CCCP.
1. d4 g5 2. Bxg5 Nc6 3. Bxe7 Kxe7 4. d5 Kd6
5. dxc6+ Kc5 6. Qd6+ Kc4 7. e4++

% miraculous win for "generous" strategy as black:
% playing against "suicide king". White forces black
% to mate him:
1. a4 b5 2. d3 Nh6 3. Kd2 Ng8 4. Kc3 Nh6
5. Kd4 c6 6. Ke5 f5 7. a5 Qb6 8. Bg5 Qe3+
9. Bxe3 Na6 10. f3 Nc5 11. Ra2 Ne4 12. f4 Nd2
13. Bf2 Nf3+ 14. exf3 Ng4+ 15. Kxf5 g5 16. Nh3 e5
17. Bh4 Ba3 18. b4 e4 19. Nf2 c5 20. c4 Ba6
21. g3 Rc8 22. Be2 Kf7 23. fxe4 d5 24. Qc1 dxe4
25. Rg1 bxc4 26. Rd1 cxd3 27. Qxc5 Rhf8 28. Nd2 Bc4
29. Kxg5 Rc7 30. Bf3 Ke8 31. Qe5+ Kf7 32. Qe6+ Kg7
33. Qb6 Rxf4 34. Nb3 Bb5 35. Qb7 d2 36. Nh3 Bc1
37. Qxa7 Be2 38. Nd4 h5 39. Ra1 Nf2 40. Raxc1 Nh1
41. Rf1 d1=Q 42. Qxc7+ Kh8 43. Kg6 Bc4 44. Bxd1 e3
45. Qf7 Bb3 46. Rf3 Nf2 47. Rc6 e2 48. Rc7 Rf5
49. Rc4 Rf4 50. Nb5 e1=Q 51. Rc2 Qc3 52. g4 Rf5
53. Ng1 Qc7 54. Rcc3 Nd3 55. Rc5 Nf2 56. Nc3 Ba2
57. Qh7+

% worstfish (black) somehow beating suicide king
% white forces the mate. Note that worstfish
% does assume its opponent will pick good moves...
1. Nf3 g5 2. b4 Na6 3. Nxg5 Nc5 4. b5 f6
5. c4 Na6 6. Nf7 Rb8 7. d3 h6 8. Kd2 Nb4
9. Kc3 Rh7 10. Kd4 Rg7 11. Kc5 Nd5 12. Nxh6 Nb4
13. Na3 Rg3 14. Qc2 Bg7 15. f4 a6 16. Qb1 Bh8
17. Qb2 f5 18. Qf6 Rxg2 19. Qh4 Rxe2 20. Bg2 e5
21. Bd2 Bf6 22. Nf7 Bh8 23. Qf2 c6 24. Kd6 Re4
25. h4 Qc7+ 26. Kxc7 Nxa2 27. Raf1 c5 28. Nh6 Ke7
29. Qd4 Ke6 30. Rc1 Bg7 31. Be1 Ra8 32. Rc2 Nc3
33. Ra2 Ne7 34. Qxe4 Bf8 35. Kd8 Nxb5 36. Qxf5+ Kd6
37. Qxd7+ Bxd7++


% Fate-based algorithms.
% The deterministic version is fairly boring, since the
% highest probability states are similar to the starting
% position (this might not be true for "survivalist"
% or "fatalist"? check?)

% weighted random is almost the same as random, because
% the weights tend to be very similar when treated
% additively. They perform slightly different from
% random, as so (101.5 million games):

%  player	elo	w/l/d
%  dangerous	986.29	1374577/1501526/15683897
%  popular	997.34	1360053/1502995/15696952
%  random_move	1000.79	1540798/1337668/15681534
%  rare	1005.85	1554600/1326707/15678693
%  safe	1009.72	1341030/1502162/15716808

% then with normalization:
%  player	elo	w/l/d
%  popular	990.93	25756/43561/346683
%  safe	993.49	28082/40769/347149
%  dangerous	996.60	30063/37451/348486
%  rare	1005.82	50656/27390/337954
%  random_move	1013.15	41747/27133/347120


// Good site for some computer chess programs with ELO:
// http://www.computerchess.org.uk/ccrl/404/


\nocite{elo1978rating}
\nocite{topple}
// citation for knights distance:
\nocite{miller2013counting}

\bibliography{chess}{}
\bibliographystyle{plain}

\end{document}
